\documentclass[12pt]{article}
\usepackage{amsmath}
\usepackage{breakurl}
\usepackage[utf8]{inputenc}
\usepackage[titletoc,toc,title]{appendix}
\usepackage{float}
\usepackage{enumitem}
\usepackage{hyperref}
\usepackage{polski}

% images
\usepackage{graphicx}
\DeclareGraphicsExtensions{.pdf,.png,.jpg}
\graphicspath{ {./materials/} }

% listings
\usepackage{listings}
\usepackage[usenames,dvipsnames]{color}
\definecolor{listinggray}{gray}{0.9}
\definecolor{lbcolor}{rgb}{0.95,0.95,0.95}
\lstset{
backgroundcolor=\color{lbcolor},
    tabsize=4,    
%   rulecolor=,
    language=C,
        basicstyle=\scriptsize,
        aboveskip={1.5\baselineskip},
        columns=fixed,
        showstringspaces=false,
        extendedchars=false,
        breaklines=true,
        prebreak = \raisebox{0ex}[0ex][0ex]{\ensuremath{\hookleftarrow}},
        frame=single,
        numbers=left,
        showtabs=false,
        showspaces=false,
        showstringspaces=false,
        identifierstyle=\ttfamily,
        keywordstyle=\color[rgb]{0,0,1},
        commentstyle=\color[rgb]{0.026,0.112,0.095},
        stringstyle=\color[rgb]{0.627,0.126,0.941},
        numberstyle=\color[rgb]{0.205, 0.142, 0.73},
%        \lstdefinestyle{C++}{language=C++,style=numbers}’.
}
\lstset{
    backgroundcolor=\color{lbcolor},
    tabsize=4,
  language=C++,
  captionpos=b,
  tabsize=3,
  frame=lines,
  numbers=left,
  numberstyle=\tiny,
  numbersep=5pt,
  breaklines=true,
  showstringspaces=false,
  basicstyle=\footnotesize,
%  identifierstyle=\color{magenta},
  keywordstyle=\color[rgb]{0,0,1},
  commentstyle=\color{OliveGreen},
  stringstyle=\color{red}
  }
  
\lstdefinelanguage{JavaScript}{
  keywords={break, case, catch, continue, debugger, default, delete, do, else, finally, for, function, if, in, instanceof, new, return, switch, this, throw, try, typeof, var, void, while, with},
  morecomment=[l]{//},
  morecomment=[s]{/*}{*/},
  morestring=[b]',
  morestring=[b]",
  sensitive=true
}
  
\title{Indeksowanie przestrzenne źródeł danych strumieniowych w silniku strumieniowej materializowanej listy agregatów}
\author{
	  Pelczar Piotr\\
	  \small{\texttt{piotpel817@student.polsl.pl}}
	  \\[3ex]
	  Sikora Paweł\\
	  \small{\texttt{pawesik788@student.polsl.pl}}
	}
\usepackage{datetime}
\newdate{date}{09}{09}{2014}
\date{\displaydate{date}}
 
\begin{document}
\maketitle
 
\begin{abstract}
This paper describes test.
\end{abstract}

\renewcommand{\contentsname}{Contents}

\newpage
\tableofcontents

\newpage

\section{Wstęp}
\label{sec:intro}

Projekt zadany w ramach przedmiotu Teoria Przestrzeni Danych i Algorytmów zakładał:

\begin{itemize}[noitemsep]
  \item zapoznanie się z metodami indeksowania źródeł danych strumieniowych, na przykładzie danych pomiarowych pochodzących ze stacji paliw,
  \item stworzenie generycznego generatora danych emitującego zdarzenia na podstawie założonego modelu (charakterystyce), opisanego w sekcji \ref{sec:eventgenerator} odpowiadającemu strumieniowemu źródle danych \ref{stream-query-gorawski},
  \item zaproponowanie rozwiązania implementacji zmaterializowanej listy agregatów dla przeprowadzania obliczeń na strukturach danych w sekcji \ref{sec:implementation}.
\end{itemize}

Dodatkowo zostały rozpoznane opensourceowe rozwiązania w kontekście ich potencjalnego wykorzystania, część wiedzy zostało zebrane w sekcji \ref{sec:solutions}.

Praca oraz kody programów zostały opublikowane w serwisie GitHub:\\
\url{https://github.com/athlan/polsl-tpdia-event-generator}

Oprócz teoretycznej dywagacji, wykonano szereg testów, w tym praktyczne zastosowanie algorytmu MapReduce do agregacji dziennych statystyk dotyczących wpisów wybranych użytkowników umieszczanych w serwisie Twitter (podsekcja \ref{sec:mapreduce-implementation-twitter}). Celem zebrania statystyk jest określenie zasięgu użytkownika oraz probie ocenienia, jaki ma wpływ na innych użytkowników (ang. \emph{influcene}).

W serwisie Twitter użytkownicy zamieszczają około 6.000 wiadomości na sekundę, co daje ponad 518 milionów wiadomości dziennie. Obserwujemy tylko mały wycinek wiadomości, natomiast już do tak małej skali trzeba zaprzęc algorytm MapReduce. To obrazuje, ile danych dziennie danych gromadzonych jest w całym serwisie i z jakimi problemami ekstrakcji danych zmaga się firma Twitter.


\section{Generator zdarzeń}
\label{sec:eventgenerator}

Test.


\section{Zmaterializowana lista agregatów}

Aby poradzić sobie z przetwarzaniem większej ilości danych w ramach systemu zarządzania bazą danych stosuje się podejścia umożliwiające skalowanie horyzontalne oraz agregowanie danych. Dla rozwiązań z dużą ilością danych istnieją konkurencyjne rozwiązania do przeprowadzania obliczeń dla zapytań kierowanych bezpośrednio do silnika relacyjnej lub nierelacyjnej bazy danych.

\subsection{Algorytm Map-Reduce}

MapReduce to podejście do przetwarzania ogromnych zbiorów danych w rozproszonych klastrach komputerowych\cite{google-map-reduce}. Stworzone przez Google rozwiązanie ma przewagę nad przetwarzaniem danych wewnątrz systemu bazodanowego, że umożliwia wykonywanie obliczeń na węzłach (w szczególności na klastrze komputerowym).

Podejście opłaca się, jeżeli infrastruktura zapewnia szybki przepływ informacji oraz wysoką odporność na błędy.

Model zakłada dwie fazy opisane w podsekcjach.

\subsubsection{Faza Map}

Główny węzeł przyjmuje dane i definiuje podproblemy, które są wysyłane do innych węzłów. Inne węzły mogą ponownie wykonać fazę map, zgodnie z logicznym połączeniem, dostępnymi zasobami, na zasadzie struktury drzewiastej. Gdy problem zostaje rozwiązany, informacja z rozwiązaniem pokonuje tę samą ścieżkę z powrotem, aż do głównego węzła.

\subsubsection{Faza Shuffle}

Faza Shuffle przydziela wartości węzłom rozpoczynającym fazę Reduce klucze, nad którymi powinny pracować. Zapewnia, że praca jest równo rozłożona pomiędzy węzły.

\subsubsection{Faza Reduce}

Główny węzeł oraz podwezły, które rozpoczęły fazę map zbierają odpowiedzi ze wszystkich podproblemów i formują je, aby udzielić odpowiedzi. W końcu, odpowiedzi trafiają do głównego węzła, który odpowiada wynikiem operacji.

\subsubsection{Implementacja Map-Reduce na przykładzie bazy danych MongoDB}



\subsection{Agregowanie strumieniowych danych}




\section{Przegląd dostępnych rozwiązań dostępnych na rynku}
\label{sec:solutions}

\subsection{Hadoop}
\label{sec:solutions:hadoop}
Hadoop to darmowy framework rozwijany przez firmę Apache. Jest to framework oparty o język Java, który wspiera przetwarzanie  dużych zbiorów danych w rozproszonym środowisku obliczeniowym. Hadoop umożliwia uruchomienie aplikacji na systemie z tysiącami węzłów obejmujących tysiące terabajtów danych. Jego rozproszony system plików umożliwia szybkie przesyłanie danych pomiędzy poszczególnymi węzłami oraz działa nieprzerwanie w przypadku awarii węzła. To podejście zmniejsza  ryzyko wystąpienia awarii systemy, nawet jeśli znacząca ilość węzłów ulegnie awarii. Obecnie Hadoop zawiera:

\begin{itemize}[noitemsep]
\item jądro Hadoop
\item MapReduce
\item rozproszony system plików Hadoop (HDFS
\end{itemize}

Największymi zaletami Hadoop są:
\begin{itemize}[noitemsep]
\item niezawodność (nawet w przypadku awarii węzłów system dalej działa)
\item skalowalność
\item rozproszenie
\end{itemize}

\subsection{Hive}
\label{sec:solutions:hive}
Apache Hive to projekt, który umożliwia tworzenie kwerend i zarządzanie dużym zbiorem danych znajdujących się w pamięci rozproszonej. Hive wykorzystuje prosty język zapytań podobny do języka SQL. Język ten nazwany został QL. Język ten umożliwia tworzenie zapytań podobnych do zapytań SQL i umożliwia tworzenie własnych funkcji mapujących i redukujących przez użytkowników, którzy znają framework Map/Reduce. Język QL wykorzystywany przez Hive może być rozszerzony przez funkcje skalarne(UDF), agregujące (UDAF) i funkcje tablicowe (UDTF).

Apache Hive posiada:

\begin{itemize}[noitemsep]
\item narzędzia ułatwiających proces ETL,
\item mechanizm do przetwarzania różnych formatów danych,
\item zapytania wykonywane przy pomocy Map/Reduce
\end{itemize}

Hive nie oferuje przetwarzania zapytań w czasie rzeczywistym i aktualizacji wierszy o niskim narzucie czasowym. Hive najlepiej wykorzystać do dużych zbiorów danych, które ciągle napływają. Największymi zaletami Hive są:

\begin{itemize}[noitemsep]
\item skalowalność (skaluje się poprzez dodanie dodatkowych dynamicznych klastrów do Hadoop)
\item rozszerzalność ( MapReduce framework, UDF/UDA/UDTF)
\item tolerancja błędów
\end{itemize}

Hive zawiera HCatalog i WebHCat. HCatalog jest warstwą zarządzającą tabelami i pamięcią dla Hadoop, która możliwa użytkownikowi korzystać z różnych narzędzi do przetwarzania danych (np. MapReduce) w celu łatwiejszego odczytu i zapisu danych. WebHCat oferuje usługi, które umożliwiają uruchomienia na Hadoop'ie MapReduce, zadań Hive lub wykonać operację metadanową za pomocą protokołu HTTP(REST).

\subsection{Spark (Stream Spark)}
\label{sec:solutions:spark}

Apache Spark to projekt, którego częścią jest Apache Spark Streaming, czyli framework, który umożliwia przetwarzanie strumieniowych, stale napływających o dużej częstotliwości danych w czasie rzeczywistym. Dane mogą pochodzić z różnych źródeł, takich jak Kafka, Flume, ZeroMQ, czy natywne gniazda TCP (czyli implementacja zaproponowana w \ref{sec:eventgenerator-eventreceiver}). Przetworzone dane mogą być składowane na dysku, w bazach danych, wysyłane na szyny danych. Spark dostarcza również wbudowane algorytmy uczenia maszynowego (ang. \emph{machine learning}) oraz analizowania grafów (ang. \emph{graph processing}). Dane są definiowane pojęciem RDD\cite{manual-apache-spark-streaming} i nie zmieniają swojego stanu (\emph{immutable state}).

\begin{figure}[h!]
  \centering
    \includegraphics[scale=0.75]{spark-streaming-arch.png}
  \caption{Zasada działania Apache Spark Stream}
  \label{fig:spark-streaming-arch}
\end{figure}

Apache Spark Streaming umożliwia również obliczenia na przesuwnym oknie RDD:

\begin{figure}[h!]
  \centering
    \includegraphics[scale=0.75]{spark-streaming-dstream-window.png}
  \caption{Przesuwne okno czasowe w Apache Spark Streaming}
  \label{fig:spark-streaming-dstream-window}
\end{figure}

Wykorzystanie Spark Streaming umożliwia wykonywanie operacji na strumieniu danych znanych z MapReduce oraz strumieniowych baz danych, w szczególności:

\begin{itemize}[noitemsep]
  \item map
  \item reduce
  \item filter
  \item transform
  \item union - złączenie źródeł danych
  \item count oraz countByValue
\end{itemize}


\section{Implementacja}
\label{sec:impl}

\subsection{Zmaterializowana lista agregatów}
\subsubsection{Pojęcie agregatu}
\label{sec:impl-mal-aggregate}

Na potrzeby projektu zostało zdefiniowane pojęcie agregatu\cite{mal-lru-gorawski} jako obiektu który zawiera jednostkę informacji skojarzoną z danym kluczem. Inaczej mówiąc, jednostka informacji jest rozumiana jako zgrupowana wartość według klucza.

Rozpatrujemy listę zmaterializowanych agregatów dla przestrzennego klucza temporalno-czasowego, których kluczem jest identyfikator miejsca (\emph{venueId}) oraz znacznik czasowy (\emph{timestamp}) wyrównany do okna czasowego (np minuty, godziny, dnia, miesiąca, roku)\footnote{Poprzez wyrównanie stempla czasowego do godziny mamy na myśli stempel czasowy pierwszej sekundy danej godziny, dla roku pierwszą sekundę pierszego dnia itd.}.

Zostały zdefiniowane następujące struktury w języku Java:

\begin{itemize}[noitemsep]
  \item \emph{IAggregateKey} - klucz agregatu, obiekt immutable (klucz może być złożony),
  \item \emph{IAggregateValue} - wartość agregatu, obiekt immutable (wartość agregatu może składać się z wielu pól),
  \item \emph{Aggregate} - zmaterializowany agregat, obiekt immutable (połączenie klucza z wartością).
\end{itemize}

\lstinputlisting[language=java]{../MaterializedAggregationFramework/src/pl/polsl/tpdia/mal/IAggregateKey.java}

\lstinputlisting[language=java]{../MaterializedAggregationFramework/src/pl/polsl/tpdia/mal/IAggregateValue.java}

\lstinputlisting[language=java]{../MaterializedAggregationFramework/src/pl/polsl/tpdia/mal/Aggregate.java}

\subsection{Mapowanie danych na agregaty}

Każda porcja informacji (krotka) napływająca strumieniowo, która trafia do systemu ma ściśle okrśloną strukturę, tj. klucz oraz zestaw wartości. Jednocześnie każda krotka sama w sobie jest agregatem (wartością atomową). Zdefiniowane zostały interfejsy pozwalające mapować agregaty na inne agregaty nazywane Mapperami (\emph{IAggregateMapper}).

Mappery przekształcają agregat w agregat o tych samych wartościach, lecz o zmienionym kluczu. Intencją jest zmmieniona granulacja klucza agregatu, na przykład implementacją mapera może być przekształcenie klucza krotki na wyrównany co do miesiąca, aby produkotwać krotki o wartościach pogrupowanych po miesiącu.

\lstinputlisting[language=java]{../MaterializedAggregationFramework/src/pl/polsl/tpdia/mal/IAggregateKey.java}

\subsection{Mappery oraz struktura agregatów}

Mapery mogą zostać połączone w łańcuch (\emph{Mapper Chain}), tak, aby dana krotka została zamapowana na wszystkie możliwe (w zależności od kontekstu, w jakim został zbudowany łańcuch) granulacje. Przykładowo istnieje możliwość połączenia Mapperów w łańcuch tak, aby wartość atomowa została przekształcona na agregat minutowy, następnie agregat minutowy na godzinowy, następnie agregat godzinowy na dniowy, itd. Produkowanie agregatów w łańcuchu prezentuje schemat \ref{fig:impl-mapping-chain}

\begin{figure}[h!]
  \centering
    \includegraphics[scale=0.65]{impl-mapping-chain.png}
  \caption{Zasada łączenia Mapperów w łańcuch}
  \label{fig:impl-mapping-chain}
\end{figure}

\lstinputlisting[language=java]{../MaterializedAggregationFramework/src/pl/polsl/tpdia/mal/IAggregateMapper.java}

Agregaty logicznie ułożone są w kompozycję przypominającą dynamicznie tworzoną strukturę drzewiastą wg klucza. Fizycznie nie ma między nimi żadnych powiązań (referencji), a źródło danych, w którym utrwalane są agregaty, może być odpytywane punktowo.

\subsection{Tworzenie agregatów i procesor zapytań}

Każda porcja informacji, kótra trafia do systemu, zamieniana jest na agregat i iterpretowana jako wartość atomowa. Operacja ta ma na celu narzucenie struktury porcji informacji, czyli klucza oraz zestawu wartości. W ten sposób tworzone są pierwsze agregaty.

Zapytania zadane systemowi są przekazywane \emph{Procesorowi zapytań}, który rozbijane je na 4 fazy:

\begin{itemize}[noitemsep]
  \item analiza semantyczna zapytania,
  \item na podstawie łańcucha maperów zaplanowanie wszystkich możliwie największych agregatów, z których można złożyć odpowiedź
  \item kaskadowe zbieranie wyników ze wszystkich istniejących agregatów oraz (jeżeli nie istnieją) tworzenie nowych, wcześniej zaplanowanych agregatów, z wyników pośrednich,
  \item złączenie wyników z możliwie największych agregatów i udzielenie odpowiedzi
\end{itemize}

Agregaty mogą być tworzone w dowolnych momentach, a w szczególności:

\begin{itemize}[noitemsep]
  \item w momencie pojawienia się krotki w systemie\footnote{Podobnie jak struktury indeksujace w bazach daych} lub,
  \item przy pierwszym zapytaniu lub,
  \item w zależności od charakterystyki i obciążenia systemu (np jest więcej zapisów, niż odczytów), obliczanie agregatów może zostać opóźniane
\end{itemize}

Za wyliczenie agregatów odpowiedzialny jest mechanizm Reducera, który przyjmuje jako argumenty wjściowe klucz, po którym agregowane dane oraz listę wartości do agregacji. Jako wynik zwraca nowy agregat.

\lstinputlisting[language=java]{../MaterializedAggregationFramework/src/pl/polsl/tpdia/mal/IAggregateReducer.java}

\subsubsection{Sposób wyznaczania tworzenia agregatów przez procesor zapytań}

Gdy do systemu trafia zapytanie zakresowe po kluczu \emph{IAggregateKey} o wyliczenie wartości bazjąych na \emph{IAggregateValue} procesor zapytań analizuje zakres zapytania i na podstawie łańcucha mapperów przygotowuje listę wszystkich możliwie największych agregatów, które mogą zostać wykorzystane do zwrócenia wyniku zapytania.

\textbf{Przyład}

Przyjmujemy założenie, że agregacja informacji następuje po miejsach (\emph{venueId}) oraz po dacie wyrównanej do: roku, miesiąca, dnia, godziny, minuty. Jeżeli zostaje zadane zapytanie zakresowe od 2014-09-24 do 2014-11-05, wyznaczane są następujące możliwie największe agregaty:

\begin{itemize}[noitemsep]
  \item dzienne od 2014-09-24 do 2014-09-30,
  \item miesięczne od 2014-10-01 do 2014-10-31,
  \item dzienne od 2014-11-01 do 2014-11-05,
\end{itemize}

\subsubsection{Drążenie danych - wyznaczanie nieistniejących agregatów}
\label{sec:impl-mal-computing-aggregates}

Jeżeli agregaty, o które system został zapytany jeszcze nie istnieją w trwałym źródle danych, dekomponowane są na możliwie największe podagregaty (\emph{drążenie danych}). Jeżeli któreś podagregaty są już wcześniej obliczone, nie ma potrzeby ich ponownej dekompozycji, natomiast dla podagregatów, które nie są policzone proces powtarza się, aż proces dekompozycji sięgnie danych źródłowych, które są agregatami atomowymi.

Schemat działania zobrazowany jest na rysunku \ref{fig:impl-aggregate-chunking}.

\begin{figure}[h!]
  \centering
    \includegraphics[scale=0.65]{impl-aggregate-chunking.png}
  \caption{Dekomponowanie agregatów oraz drążenie danych. Zielone kwadraty oznaczają obliczone pola, szare nieistniejące, pomarańczowym kolorem oznaczone zostały agregaty atomowe, czyli dane źródłowe}
  \label{fig:impl-aggregate-chunking}
\end{figure}

\subsection{Aktualizowanie agregatów po pojawieniu się opóźnionej krotki}

Dane moga pojawiać się w systemie z opóźnieniem\cite{stream-processing-streamsql}\footnote{\emph{Rule 3: Handle Stream Imperfections (Delayed, Missing and Out-of-Order Data)}}. Przewidziano mechanizmy zapewniające prawidłową prace systemu podczas, gdy agregaty zostały już wyliczone, natomiast pojawienie się nowej danej dezaktualziuje wcześniej zagregowane dane, które pasują do kluczy agregatów wyznaczonych przez mappery.

Gdy nowa krotka trafia do systemu, wyliczany jest jej agregat atomowy, a wraz z nim, zgodnie z łańcuchem mapperów po kolei wyznaczane są wszystkie klucze możliwych do stowrzenia agregatów, kaskadowo do góry z logiczną strukturą. Potencjalnie istniejące agregaty odpowiadające wyznaczonym kluczom są odszukiwane w trwałym źródle danych i oznaczane jako nieważne (\emph{invalid}).

Zaprezentowana w sekcji \ref{sec:impl-mal-aggregate} struktura agregatu posiada pole określające czy agregat jest aktualny. Jeżeli okaże się, że dane agrekaty ulegną dezaktualizacji, można podjąć kroki opisane w sekcji \ref{sec:impl-mal-computing-aggregates} i drążyć agregaty ponownie wykorzystując jednak wszystkie policzone i ważne agregaty na wszytkich poziomach struktury.

Schemat działania zobrazowany jest na rysunku \ref{fig:impl-aggregate-chunking-rebuild}.

\begin{figure}[h!]
  \centering
    \includegraphics[scale=0.65]{impl-aggregate-chunking-rebuild.png}
  \caption{Ponowne przeliczanie nieaktualnych agregatów z wykorzystaniem już wyloczonych danych. Czerwonym kolorem zostały zaznaczone agregaty, które zostały zdezaktulizowane poprzez pojawienie się danych z opóźnieniem. Zielone kwadraty oznaczają obliczone pola, szare nieistniejące, pomarańczowym kolorem oznaczone zostały agregaty atomowe, czyli dane źródłowe.}
  \label{fig:impl-aggregate-chunking-rebuild}
\end{figure}

\subsection{Przykładowa implementacja}

Poniżej przedstawiono przykładową definicję modelu danych podzących ze stacji paliw na podstawie struktur zdefiniowanych w rozdziale \label{sec:impl-mal-aggregate}. Zostały zaproponowane proty mapper oraz reducer.


\lstinputlisting[language=java]{../MaterializedAggregationFramework/src/pl/polsl/tpdia/mal/impl/TuppleEntity.java}

\lstinputlisting[language=java]{../MaterializedAggregationFramework/src/pl/polsl/tpdia/mal/impl/TuppleKey.java}

\lstinputlisting[language=java]{../MaterializedAggregationFramework/src/pl/polsl/tpdia/mal/impl/TuppleValue.java}

\lstinputlisting[language=java]{../MaterializedAggregationFramework/src/pl/polsl/tpdia/mal/impl/TuppleToAggregateMapper.java}

\lstinputlisting[language=java]{../MaterializedAggregationFramework/src/pl/polsl/tpdia/mal/impl/TuppleToAggregateMapperGranulity.java}

\lstinputlisting[language=java]{../MaterializedAggregationFramework/src/pl/polsl/tpdia/mal/impl/TuppleMaterializedAggregateReducer.java}

Powyższe struktury moga zostać odzworowane jako encje bazodanowe za pomocą biblioteki umożliwiającej mapowanie obiektowo-relacyjne, na przykład za pomocą Hibernate. Z uwagi na fakt, iż przedstawiana jest jedynie koncepcja rozwiązania, zgodnie z ideą tworzenia architektury heksagonalnej (nie pożenionej z żadnym konkretnym frameworkiem), struktry pozostawnioe zostały jako zwykłe klasy Java.

Mapper i Reducer może bezpośrednio działać w oparciu o Apache Spark Stream.


\newpage
\begin{thebibliography}{9}
\addcontentsline{toc}{section}{Bibliografia}

\bibitem{cormen}
  Thomas H. Cormen,
  \emph{Introduction To Algorithms}.
  The MIT Press,
  2nd Edition,
  2003.

\end{thebibliography}

\begin{appendices}
	\section{App test 1}
	\label{app:appTest1}

\end{appendices}

\end{document}