\newpage
\section{Generator zdarzeń}
\label{sec:eventgenerator}

\subsection{Opis generatora zdarzeń}
Program przy pomocy wykresu zamieszczonego w bmp tworzy tablicę z danymi, która zawiera informację o ilości tankowań w danym dniu. Program dzieli dobrę na ilość okresów czasów, ich ilość jest równa długości wykresu, natomiast wysokość odpowiada ilości tankowań w tym okresie. Następnie dane przekazywane są do programu, który emituje zdarzenia z danego okresu czasu w równych odstępach. Aplikacja została napisana w sposób by możliwe było podpięcie dużej ilości klientów emitujących dane. 
\subsection{Model danych}
Modelem danych w programie przechowującym informację i ilości tankowań w danym dniu jest tablica jednowymiarowa. Pojedyncza wartość tabeli zawiera informację o ilość zdarzeń w danym okresie czasu. Długość przedzialów czasowy obliczana jest przez podzielenie doby przez długość tabeli ( czyli przez ilość przedziałów czasowych). Długość przedziałów czasowych nie jest ustawiana na sztywno lecz dostosowuje się do ilości dostarczonych w tabeli danych. 
\subsection{Opis danych wejściowych}
W danym przykładzie danie pobieranie są z wykresu zapisanym w pliku BMP. Długość wykresu wyznacza ilość przedziałów czasowym, natomiast odległość pierwszego napotkanego czarnego piksela w danej kolumnie mierzona od dolnej granicy wykresu odpowiada  ilości zdarzeń w danym okresie czasu. W przypadku wystąpienia dwóch lub więcej wartości w kolumnie uwzględniona zostanie mniejsza wartośc.
\subsection{Struktura programu}
\subsubsection {ModelDataLoader}
Moduł "ModelDataLoader" odpowiada za wywołanie funkcji wczytującej dane wejściowe. W poniższym przykładzie wczytuje kod w języku Java, który przygotowuje model danych na podstawie wykresu zapisanego w pliku bmp. Moduł umożliwa prostą zmianę sposobu wczytywanie danych, np. można napisac program przygotowujący dane na podstawie pliku txt, lub akrusz kalkulacyjnego Excel.
\lstinputlisting[language=java]{./materials/DumpPlotDataFromBmp.java}
fragment pliku DumpPlotDataFromBmp.java

\subsubsection{EventEmitter}
Moduł EventEmitter na podstawie modelu danych przekazanych przez "ModelDataLoader" dokonuje obliczenia długości przedziałów czasowych. Dla każdego przedziału czasowego oblicza odstęp czasowy pomiędzie zdarzeniami i dla każdego z nich wywołuje funkcję eventOccurenceCallback, która wystawia na lokalną szynę danych zdarzenie "eventOccurence". W naszej implementacji w przypadku wystąpnienia zdarzenia "eventOccurence" zostaje wywołana funkcja serwera tcp, który rozpropaguje infromację do modułów EventsReceiver. Serwer TCP zaimplementowany został na porcie 3005.
\lstinputlisting[language=JavaScript]{./materials/EventEmitter.js}
fragment pliku EventEmitter.js

\subsubsection{EventReceiver}
Moduł "EventReceiver" ma za zadanie nasłuchiwać i odbierać informację o pojawieniu się nowego zdarzenia. W naszej implementacji "Event Receiver" jest napisany jako klient TCP, który oczekuje na zdarzenia modułu "EventEmitter"
\lstinputlisting[language=JavaScript]{./materials/sample-client.js}
fragment pliku sample-client.js
Istnieje możliwość podłączenie większej ilość modułów "EventReceiver".


\subsection{Uruchomienie generatora zdarzeń}
\subsubsection{Wymagania}
Do uruchomienia programu niezbędne są następujące narzędzie:
\begin{itemize}[noitemsep]
  \item Java (JRE 1.7)
  \item node.js v.0.10.15
\end{itemize}
\subsubsection{Uruchomienie programu z implementacją modułu EventEmitter}
W celu uruchomienia programu, który implementuje moduł EventEmitter należy uruchomić konsolę systemową w folderze z generatorem zdarzeń i wpisać następujące komendy:
\begin{itemize}[noitemsep]
  \item cd EventEmitter
  \item index.js
\end{itemize}
\subsubsection{Uruchomienie programu z implementacją modułu EventReceiver}
W celu uruchomienia programu, który implementuje moduł EventReceiver należy uruchomić konsolę systemową w folderze z generatorem zdarzeń i wpisać następujące komendy:
\begin{itemize}[noitemsep]
  \item cd EventEmitter
  \item sample-client.js
\end{itemize}


\subsection{Użyte technologie}
\begin{itemize}[noitemsep]
  \item Java
  \item Node.js
\end{itemize}