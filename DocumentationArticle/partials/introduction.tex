\section{Wstęp}
\label{sec:intro}

Projekt zadany w ramach przedmiotu Teoria Przestrzeni Danych i Algorytmów zakładał:

\begin{itemize}[noitemsep]
  \item zapoznanie się z metodami indeksowania źródeł danych strumieniowych, na przykładzie danych pomiarowych pochodzących ze stacji paliw,
  \item stworzenie generycznego generatora danych emitującego zdarzenia na podstawie założonego modelu (charakterystyce), opisanego w sekcji \ref{sec:eventgenerator},
  \item zaproponowanie rozwiązania implementacji zmaterializowanej listy agregatów dla przeprowadzania obliczeń na strukturach danych w sekcji \ref{sec:implementation}.
\end{itemize}

Dodatkowo zostały rozpoznane opensourceowe rozwiązania w kontekście ich potencjalnego wykorzystania, część wiedzy zostało zebrane w sekcji \ref{sec:solutions}.

Praca oraz kody programów zostały opublikowane w serwisie GitHub:\\
\url{https://github.com/athlan/polsl-tpdia-event-generator}
