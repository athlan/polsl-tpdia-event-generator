\section{Zmaterializowana lista agregatów}

Aby poradzić sobie z przetwarzaniem większej ilości danych w ramach systemu zarządzania bazą danych stosuje się podejścia umożliwiające skalowanie horyzontalne oraz agregowanie danych. Dla rozwiązań z dużą ilością danych istnieją konkurencyjne rozwiązania do przeprowadzania obliczeń dla zapytań kierowanych bezpośrednio do silnika relacyjnej lub nierelacyjnej bazy danych.

\subsection{Algorytm Map-Reduce}

MapReduce to podejście do przetwarzania ogromnych zbiorów danych w rozproszonych klastrach komputerowych\cite{google-map-reduce}. Stworzone przez Google rozwiązanie ma przewagę nad przetwarzaniem danych wewnątrz systemu bazodanowego, że umożliwia wykonywanie obliczeń na węzłach (w szczególności na klastrze komputerowym).

Podejście opłaca się, jeżeli infrastruktura zapewnia szybki przepływ informacji oraz wysoką odporność na błędy.

Model zakłada dwie fazy opisane w podsekcjach.

\subsubsection{Faza Map}

Główny węzeł przyjmuje dane i definiuje podproblemy, które są wysyłane do innych węzłów. Inne węzły mogą ponownie wykonać fazę map, zgodnie z logicznym połączeniem, dostępnymi zasobami, na zasadzie struktury drzewiastej. Gdy problem zostaje rozwiązany, informacja z rozwiązaniem pokonuje tę samą ścieżkę z powrotem, aż do głównego węzła.

\subsubsection{Faza Shuffle}

Faza Shuffle przydziela wartości węzłom rozpoczynającym fazę Reduce klucze, nad którymi powinny pracować. Zapewnia, że praca jest równo rozłożona pomiędzy węzły.

\subsubsection{Faza Reduce}

Główny węzeł oraz podwezły, które rozpoczęły fazę map zbierają odpowiedzi ze wszystkich podproblemów i formują je, aby udzielić odpowiedzi. W końcu, odpowiedzi trafiają do głównego węzła, który odpowiada wynikiem operacji.

\subsubsection{Implementacja Map-Reduce na przykładzie bazy danych MongoDB}



\subsection{Agregowanie strumieniowych danych}


